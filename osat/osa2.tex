\documentclass[12pt]{article}
\usepackage{cite}

\begin{document}
\tableofcontents
\newpage
\section{Johdanto}
\section{Vertailtavat järjestelmät}
\subsection{ Järjestelmien esittely }
Tässä tutkielmassa vertaillaan verkkopalvelimien toteutusmalleja
ja niiden suorityskykyä. Verkkopalvelin on ohjelma joka vastaa useiden asiakkaiden pyyntöihin
verkkosivuilla tai verkkosovelluksissa. Pyyntöihin vastaamiseen
voi liittyä laskentaa, tietokantaoperaatioita tai muita toimenpiteitä.

Vertailuun valittiin asynkroninen tapahtumavetoinen palvelinmalli, sekä säiereservimalli.
Asynkroninen reactor-malli on kasvattanut suosiotaan pitkään verkkokehityksessä. 
Sitä vastaan vertailussa on säiereservimalli sillä se on perinteisistä 
monisäikeisistä palvelinmalleista relevantein ja siinä on onnistuttu korjaamaan
joitakin näiden mallien ongelmia.

\subsubsection{Asynkroninen tapahtumavetoinen palvelinmalli}

Asynkronisessa tapahtumavetoisessa palvelinmallissa yhdistetään
käyttöliittymäohjelmoinnista aiemmin tuttu tapahtumavetoinen malli
verkkopalvelimeen. Viimeisen viiden vuoden aikana tämä malli on
kasvattanut ymmärrettävästi suosiotaan, sillä verkkosovelluksia
käytetään yhä enemmän palveluiden käyttöliittyminä.

Mallin keskeiset artefaktit ovat tapahtumasäie ja työntekijäreservi.

Asynkronisessa ja tapahtumavetoisessa mallissa pyyntöjen alustavasta käsittelystä
vastaa tapahtumasäie.
Tapahtumasäie jakaa keskeyttävät operaatiot
työntekijäreservin säikeille.
Myös raskasta laskentaa voidaan 
ohjelmoida työntekijäreservin vastuulle.
Suurin osa laskentatyöstä
on kuitenkin tapahtumasäikeen vastuulla~\cite{pai_flash:_1999}.

Asynkronisten ja tapahtumavetoisten palvelinsovelluksien suunnittelun keskiössä
ovat vasteajan pienentäminen ja laitteistoresurssien tehokas hyödyntäminen. 
Tavoitteena on hyödyntää I/O-resursseja mahdollisimman tehokkaasti
rinnakkain ja välttää I/O:n odottamista ohjelman suorituksessa~\cite{pai_flash:_1999}.
Palvelinsovelluksissa
laitteiston verkkoliikennekapasiteetti on usein rajoittava tekijä ja tämän takia
sen tehokas hyödyntäminen on kriittistä. I/O-resurssien asynkronisella hyödyntämisellä on todettu
olevan suorituskykyä parantava vaikutus~\cite{hu_applying_1998}.




\subsubsection{Säiereservimalli}
Monisäikeisiä palvelinmalleja on useita, mutta tässä tutkielmassa keskitytään säiereservimalliin.
Monisäikeisissä palvelinsovelluksissa keskitytään hyödyntämään laitteiston ominaista
rinnakkaisuutta tehokkaasti suorittamalla jokaista pyyntöä omalla säikeellään. Suunnittelun keskeisenä
tavoitteena on käsitellä mahdollisimman monta pyyntöä ajan yksikössä. Yksittäisen pyynnön
vasteaika saattaa tässä
järjestelmässä kärsiä säikeiden hallinnoimiseen kuluvasta suoritinajasta~\cite{easton_developing_2004}.
Pääsääntöisesti nämä mallit skaalautuvat hyvin laitteistoresursseihin,
mutta jos hallinnoitavien säikeiden määrä on huomattavasti suurempi kuin laitteiston
fyysisten suorittimien määrä, aiheuttaa hallinointi ylimääräistä taakkaa.
Säiereservimalli pyrkii estämään tämän ongelman rajoittamalla
säikeiden määrän johonkin järkevään arvoon, kuten fyysisten suorittimien
määrään.
Menetelmässä pyyntöjen käsittely on hyvin eristetty toisistaan ja
yhden säikeen lukittava tapahtuma ei vaikuta muihiin säikeisiin negatiivisesti~\cite{davis_case_2017}.
Tämä edistää monisäikeisen mallin vakautta ja kykyä vastustaa hyökkäyksiä.
Tämä malli on myös ohjelmoijalle yksinkertainen sillä, pyyntöjen jakama tilatietoa
on vähän ja täten myös synkronoinnin tarve pyyntöjen välilä minimoituu~\cite{hu_applying_1998}.



\subsection{ Arviontikriteerit }
Palvelimen roolin kannalta olisi toivottavaa, että
se pystyisi käsittelemään mahdollisimman paljon pyyntöjä
ajan yksikössä. Sen tulisi pystyä myös käsittelemään pyyntöjä
luotettavasti eli lopulta kaikki pyynnöt käsitellään ja niihin vastataan.
Yksittäiselle asiakkaalle tärkein suorituskyvyn mittari on se kuinka kauan
yksittäiseen pyyntöön vastaaminen vie. Pyynnön matala vasteaika saa
palvelun käyttökokemuksen tuntumaan responsiivisemmalta.
Skaalautuvuus on tärkeää palvelinta ylläpitävälle taholle, sillä 
näin käyttäjämäärän kasvaessa voidaan suorituskykyä parantaa
lisäämällä laitteistoresursseja. Järjestelmän tulisi myös
olla resilientti hyökkäyksiä vastaan.

Näiden ajatusten ja aikaisemman tutkimuksen perusteella
tutkielmassa vertailtavia järjestelmiä arvioidaan seuraavien 
kriteerien valossa~\cite{gokhale_performance_2006}.
\begin{itemize}
    \item käsiteltyjen pyyntöjen määrä
    \item menetysprosentti
    \item skaalautuvuus laitteistoresursseihin
    \item vasteaika
    \item vakaus
\end{itemize}
\subsection{Järjestelmien arviointi kriteerien perusteella}
Molempien mallien puutteita voidaan paikata useamman laitteiston hajautetuilla 
järjestelmillä, mutta tälläiset järjestelyt jätetään vertailun ulkopuolelle.

Reactor-mallin järjestelmissä suoritinsidonnaisien
toimitusten rinnakkaistamiseen ei ole keskitetty erityistä huomiota, ja tälläisten operaatioiden
kanssa järjestelmän käsiteltyjen pyyntöjen määrä ajan yksikössä on keskimäärin heikompi
kuin monisäikeisten järjestelmien~\cite{davis_case_2017}.
Asynkronisten-mallien skaalautumien kasvaviin laitteistoresursseihin on hyvin
implementaatiosidonnaista.

Säiereservimallin skaalautuvuus suoritinresursseihin on hyvin suoraviivaista,
vaikkakin kilpailutilanteiden havaitsemiseen ja estämiseen kuluu enemmän aikaa,
jos säikeita on paljon.

Asynkronisen tapahtumavetoisen palvelimeen voi kohdistaa hyökkäyksiä, jolla pyritään 
lukittamaan tapahtumia käsittelevä säie, joka käytännössä vastaa koko ohjelman hallinnasta.
Tähän tarkoitukseen sopivat pyynnöt,
joiden käsittelyyn liittyy kompleksista säännöllisten lauseiden käsittelyä.
Osoitepolkujen selvittämiseen liittyy usein säännöllisiä lausekkeita,
joten verkkopalvelimet ja niiden
kehykset voivat olla altiitta tämänkaltaisille hyökkäyksille~\cite{davis_case_2017}.

Tälläiset hyökkäykset aiheuttaisivat ongelmia myös säiereservimallia
noudattaville järjestelmille, mutta niiden pitäisi onnistua
lukittamaan reservin jokainen säie samaan aikaan, jotta
järjestelmä ei kykenisi enää vastaamaan asiakkaiden pyyntöihin ~\cite{davis_case_2017}.

Järjestelmien ominaisuuksien perusteella voimme siis päätellä että,
säiereservimalli
tulisi pärjäämään paremmin, jos palvelimen tehtävän pullonkaula on suoritinaika.
Asynkroninen järjestelmä
taas luultavasti kykenee pienempään viiveeseen, jos I/O-kutsujen käsittely on suurin rajoittava tekijä.
\section{Järjestelmät testeissä}
Koejärjestelyssä on tärkeää seuraavat seikat

\begin{itemize}
    \item Säädettävä työmäärä pyynnön sisällöllä
    \item Kyky mitata menetysprosentti
    \item mahdollisuus muuttaa laitteistoresursseja?
    \item mahdollisuus mitata vasteaika
\end{itemize}
\section{Johtopäätökset}
\bibliography{kandi_zot}
\bibliographystyle{plain}

\end{document}
