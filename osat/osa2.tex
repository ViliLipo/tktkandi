\documentclass[12pt]{article}
\usepackage{cite}
\usepackage[finnish]{babel}
\title{Rinnakkaisuus modernissa verkkopalvelinsovelluksessa}
\author{Vili Lipo}

\begin{document}
\maketitle
\newpage
\tableofcontents
\newpage
\section*{Tiivistelmä}
Tässä tutkielmassa vertaillaan verkkopalvelimien toteutusmalleja
ja niiden suorityskykyä. Verkkopalvelin on ohjelma, joka vastaa useiden asiakkaiden pyyntöihin
verkkosivuilla tai verkkosovelluksissa. Pyyntöihin vastaamiseen
voi liittyä laskentaa, tietokantaoperaatioita tai muita toimenpiteitä.

Vertailuun valittiin asynkroninen tapahtumaohjattu reactor palvelinmalli, sekä säiereservimalli.
Asynkroninen reactor-malli on kasvattanut suosiotaan pitkään verkkosovelluskehityksessä.
Sitä vastaan vertailussa on säiereservimalli sillä siinä on
onnistuttu korjaamaan joitakin yksi-säie-yhteyttä-kohti-mallin ongelmia.
\section{Johdanto}
\section{Verkkopalvelinsovellus}
\subsection{Asiakas-palvelin malli}
Asiakas-palvelin mallissa
asiakasosa tarjoaa käyttäjälle käyttöliittymän sovellukseen. Palvelinosa taas
tarjoaa määritellyn rajapinnan
palveluita asiakas-osalle verkkoyhteyden yli~\cite{sinha_client-server_1992}.
Tieto välittyy osien välillä pyyntöinä ja vastauksina ja näin palvelinosa ja asiakasosa
muodostavat löysästi yhdistetyn järjestelmän.

Asiakasosa siis yhdistää käyttöliittymässä tapahtuvat toimet palvelimen
rajapinnan pyynnöiksi. Se voi hyödyntää pyyntöjen tulosten tallentamista
muistii. Näin tarvittavia pyyntöjä voidaan vähentää, jos sen tulos
on jo paikallisessa muistissa. Asiakasosa voi myös suorittaa
laskentaa tai muita toimenpiteitä paikallisesti~\cite{sinha_client-server_1992}.

Asiakasosat voivat olla työpöytäsovelluksia, jotka hyödyntävät käyttöjärjestelmän
ikkunointi-ja käyttöliittymäominaisuuksia~\cite{sinha_client-server_1992}.
Tämänkaltaisen asiakasohjelman toteuttamisessa voidaan käyttää mitä tahansa ohjelmointikieltä,
ja asiakasohjelman toiminnallisuutta voi laajentaa lähes rajatta.

Asiakasohjelma voi olla myös verkkosivu. 2000-luvun alussa
pelkkään staattiseen HTML-standardiin perustuvat verkkosivut pystyivät
tarjoamaan käyttöliittymiä esimerkiksi keskustelualustoille ja verkkokaupoille.
Näissä järjestelmissä valtaosa suorittamisesta on palvelimen vastuulla,
sillä sen pitää tiedonkäsittelemisen lisäksi kirjoittaa tieto
esitettävään HTML-muotoon käyttöliittymän näkymäksi~\cite{tatsubori_html_2009}. Palvelin
siis vastaa selaimen pyyntöön lähettämällä kokonaisen HTML-tiedoston.

Nykyisten selainten JavaScript suoritustuen ansiosta yhä
monimutkaisempia asiakassovelluksia voidaan toteuttaa
selaimella saavutettavilla verkkosovelluksina.
Nykyään palvelimen ja asiakkaan
välisessä tiedonvälityksessä on muodikasta käyttää REST-rajapintaa, jossa
tieto on rakenteisessa muodossa. Asiakasosa tämän jälkeen piirtää siitä
itsenäisesti käyttöliittymän näkymän. [lähde?]

Palvelin tarjoaa rajapinnan läpi palveluita asiakasohjelmille.
Palvelin ei itsenäisesti aloita yhteyttä mihinkään
asiakkaaseen vaan se odottaa niiden pyyntöjä.
Se pystyy palvelemaan useita asiakkaita, jopa
eri asiakasohjelmia, kunhan asiakasohjelmat noudattavat
palvelimen rajapintaa~\cite{sinha_client-server_1992}.

Palvelimen palvelut muodostavat järjestelmän keskeisen
sovelluslogiikan, jossa asiakaspyynnön parametreilla
suoritetaan toimintoja ja palvelin lähettää
tuloksen vastauksena~\cite{sinha_client-server_1992}.

Yleisesti palvelin vastaa järjestelmän turvallisuudesta sillä,
sen toiminnan vääristäminen on haastavampaa paha-aikeisille
toimijoille, kun taas asiakasosan toiminnan muuntelu, pyyntöjen
lähtettäminen toisella ohjelmistolla on suoraviivaisempaa.

Asiakasohjelmaa ajetaan laitteella, joka ei ole verkkosovelluksen
ylläpitäjän hallussa. Paha-aikeiset toimijat voitat käyttää tätä hyväkseen
ja tarkkailla asiakasosan lähettämiä pyyntöjä tai sen
suorituskäyttäytymistä kuten resurssien käyttöä.
Tarkkailusta selvinneiden tietojen avulla on mahdollista
kiertää asiakasosaan toteutettuja suojauksia.

Palvelinsovelluksen toimintaa vääristääkseen 
paha-aikeisten toimijoiden tulisi päästä
käsiksi sitä suorittavan järjestelmän hallintatoimintoihin, tai
löytää haavoittuvuus sen rajapinnasta. Nämä hyökkäsvektorit
ovat huomattavasti kapeampia kuin sovelluksella jota
suoritetaan hyökkääjien omalla laitteistolla.
% etsi jokin tietoturvan perusteos
Palvelin vastaa myös tunnistautumisesta sekä käyttöoikeuksien hallinnasta.
Monen palvelun toteuttamisen kannalta on kriittistä, että
tiedon käyttöoikeuksia voidaan rajata käyttäjän tunnistautumisen perusteella.

Yksinkertainen esimerkki asiakas-palvelin mallista on tietokantapalvelin,
jossa tietokannan tiedot on tallennettu palvelimelle, josta voidaan
asiakasohjelmalla tehdä verkon yli tietokantakyselyitä.

Verkkopalvelin on palvelin, jonka tehtävä on vastata Internetin
selaamiseen liittyviin pyyntöihin eli lähettämään tietyn HTTP-pyynnön
perusteella oikea HTML-tiedosto, tekstimuotoinen data-tiedosto~\cite{Berners-Lee_1994}
,tai kuva-tiedosto.
\subsection{Rinnakkaisuus palvelinympäristössä}

Rinnakkaisuudella tarkoitetaan tietojenkäsittelytieteessä useiden
eri operaatioiden suorittamista järjestelmässä samaan aikaan.
Nykyisissä palvelintietokoneissa on kymmeniä suorittimia ja niiden tehokkaaseen
hyödyntämiseen tarvitaan rinnakkaisuuden hallintaa.

On täysin tehtäväkohtaista
kuinka hyvin rinnakkaisuuden tavoitteet voidaan saavuttaa.
Tähän vaikuttaa kuinka suuri osa tehtävästä on olemukseltaan peräkkäistä.
Peräkkäiset osat ovat sellaisia, jossa ohjelman vaiheet ovat täysin
riippuvaisia toisistaan ja seuraavaan vaiheeseen ei voida mennä ennen edellisen
valmistumista. Jos suuri osa ohjelmasta on peräkkäistä, ei rinnakaistamalla
saada hyötyä. Tälläisessä tapauksessa rinnakkaisuuden hallinnasta aiheutuvat
rasitteet saattavat jopa heikentää suorituskykyä~\cite{stallings_operating_2018}.

Riippumatta tehtävästä rinnakkaisuuden saavuuttaminen on vaikeaa,
sillä se tuo ohjelmoijalle uusia haasteita, jotka eroavat tavallisesta
ohjelmoinnista. Ohjelmointivirheen riski on rinnakkaisissa osissa hyvin suuri,
koska ohjelman oikeellisuuden varmistaminen kaikissa mahdollisissa
suoritusjärjestyksissä on erittäin työlästä.
Keskeisiä ongelmia, joita rinnakkaisuus tuo ohjelmointiin on eri vaiheiden
synkronointi, tietorakenteiden suojaaminen ja lukkiutumisen hallinta.

Verkkopalvelinsovelluksessa yksi asiakkaan pyyntö http-protokollan välityksellä
muodostaa mielekkään kokonaisuuden.
Järjestelmässä on toivoittavaa, että pyyntöjä voitaisiin suorittaa
rinnakkain kaikilla järjestelmässä olevilla suorittimilla, tai muuten 
saavuttaa korkea resurssien käyttöaste.
Näin ohjelmistojärjestelmä skaalautuu saatavilla oleviin resursseihin
mielekkäällä tasolla. Käsiteltävät pyynnöt eivät yleisesti ole riippuvaisia
toisistaan ja niiden täyttäminen usein vaatii I/O-laitteiden odottamista.
Näistä syistä verkkopalvelinsovellus on hyvä kohde rinnakkaistamiselle.

Yleisesti verkkopalvelimet toteutetaan käyttäen valmista sovelluskehystä eli kirjastoa,
joka toteuttaa palvelinmallin, ja tarjoaa rajapinnat sovelluslogiikan
kehittämistä varten.

Käyttöjärjestelmät tarjoavat kehittäjälle työkaluja, joilla rinnakkaisuuden hallintaan.
Niistä tärkeimpiä ovat prosessit ja säikeet.
Prosessi yksi ohjelma, joka on ajossa järjestelmässä. Sen suoritustilatiedot, ja tunniste on
tallennettu käyttöjärjestelmässä prosessin kuvaajaan~\cite{stallings_operating_2018}.
Jokaisella
prosessilla on oma muistiavaruus, ja prosessien välinen keskustelu
tapahtuu ainoastaan käyttöjärjestelmän tarjoamien rajapintojen läpi~\cite{stallings_operating_2018}.
Prosessi siis käsitteenä mahdollistaa sen että, käyttöjärjestelmä
voi suorittaa useaa ohjelmaa samanaikaisesti ja käyttöjärjestelmän
vuoronantaja jakaa niille suoritusaikaa. Kun prosessi vaihtuu
käyttöjärjestelmä asettaa suorittimen suoritusympäristöksi
uuden prosessin kuvaajaan tallennetun ympäristön. Tätä kutsutaan kontekstin
vaihdoksi. Vanhan prosessin suoritinympäristö puolestaan talletetaan sen 
kuvaajaan~\cite{stallings_operating_2018}.

Nykyisissä käyttäjärjestelmissä voi prosessin suoritusta jakaa usealle
säikeelle. Prosessissa on silloin yksi tai useampia säikeitä.
Säikeet jakavat prosessin muistiavaruuden, sekä tiedostoresurssit.
Säikeiden kuvaajiin
tallennetaan niiden suoritustilatieto,
kuten rekisterien ja pinon arvot~\cite{stallings_operating_2018}.

Koska säikeisiin liittyy vähemmän tietoa, kuin prosessehin,
on niiden luominen ja tuhoaminen nopeampaa. Säikeiden välinen
kommunikointi on myös huomattavasti nopeampaa kuin prosessien, koska
ne voivat välittää tietoa jakamansa muistialueen läpi.
Prosessit voivat kommunikoida keskenään vain käyttöjärjestelmän 
avulla~\cite{stallings_operating_2018}.
Nopea
kommunikointi tekee rinnakkaisen ohjelman synkronointivaiheista
sujuvampia~\cite{stallings_operating_2018}. Säikeiden vaihdossa suoritinympäristö pitää vaihtaa 
samaan tapaan kuin prosessienkin kohdalla, mutta keskusmuistissa olevia
tietoja ei tarvitse vaihtaa, sillä saman prosessin säikeet jakavat
muistin.

Ytimentasonsäikeet ovat säikeitä joiden aikatauluttamisesta vastaa
käyttöjärjestelmän ydin. Käyttäjätasonsäikeet ovat sellaisia säikeitä,
joiden aikatauluttamisesta vastaa jokin käyttäjätason ohjelma kuten
säiekirjasto. Käyttäjätason aikatauluttamisen puutteista johtuen
kaksi saman prosessin käyttäjätason säiettä ei voi olla suorituksessa
samaan aikaan. Täten ytimentasonsäikeet palvelevat rinnakkaisuuden tavoitteita
paremmin, mutta niiden hallinnointi on raskaampaa~\cite{stallings_operating_2018}.
% Voiko taustatietoa lainata oppikirjoista huoletta ?
\subsection{Asynkroninen tapahtumaohjattu palvelinmalli}

Asynkronisessa tapahtumaohjatussa palvelimessa yhdistetään
käyttöliittymäohjelmoinnista lähtöisin oleva tapahtumaohjattu malli
verkkopalvelimeen~\cite{pai_flash:_1999}. Tapahtumaohjattu malli sopii verkkopalvelinsovellukseen,
sillä sen toiminnan rytmittää pyynnöt, jotka
kuvautuvat mallissa tapahtumiksi~\cite{schmidt_reactor:_1995}.
Prosessiehin perustuva asynkroninen AMPED\cite{pai_flash:_1999}
oli aikana jona käyttöjärjestelmät eivät yleisesti tukeneet
säikeitä tai asynkronisia kutsuja erityisen siirrettävä.
Myöhemmin näiden ominaisuuksien yleistyessä 
huomiota saaneita malleja ovat Reactor~\cite{schmidt_reactor:_1995}
ja Proactor~\cite{pyarali_proactor_1997}, joissa
rinnakkaisuus saavutetaan asynkronisilla I/O-operaatiolla
ja säikeillä. Nämä mallit ovat yleisiä tapahtumaohjattuja malleja,
joita voidaan soveltaa myös muuhunkin kuin palvelimiin.


Asynkronisten ja tapahtumaohjattujen palvelinsovelluksien suunnittelun keskiössä
ovat vasteajan pienentäminen ja laitteistoresurssien tehokas hyödyntäminen.
Tavoitteena on hyödyntää I/O-resursseja mahdollisimman tehokkaasti
rinnakkain ja välttää I/O:n odottamista ohjelman suorituksessa~\cite{pai_flash:_1999}.
Palvelinsovelluksissa
laitteiston verkkoliikennekapasiteetti on usein rajoittava tekijä ja tämän takia
sen tehokas hyödyntäminen on kriittistä. I/O-resurssien asynkronisella hyödyntämisellä on todettu
olevan suorituskykyä parantava vaikutus~\cite{hu_applying_1998}.

% miten tulee toimia kun referoin paljon tekstia jossa malli esitellään ensikerran ?
Reactor-malli sopii käyttötapauksiin, joissa pyyntöjä voi saapua 
samanaikaisesti monista lähteistä ja estävästi
pyyntöjen odottaminen näistä lähteistä olisi tehotonta~\cite{schmidt_reactor:_1995}.
Käyttötapauksen tapahtumankäsittelijöiden, tulisi
lähettää ja vastaanottaa rajoitetun kokoisia viestejä
tarvitsematta estävää I/O:ta. Näiden viestien käsittelyn
pitäisi tapahtua suhteellisen lyhyessä ajassa~\cite{schmidt_reactor:_1995}.
Myös jos kohdelaitteistolla ei ole järkevää käyttää monisäikeistä
ratkaisua pyyntöjen käsittelyn rinnakkaistamiseen tai monisäikeisyys
on toteutettu muualla järjestelmän arkkitehtuurissa, sopii Reactor-malli
erottamaan sovelluksen ydinlogiikasta tapahtumienkäsittelyn
rinnakkaistamiseen liittyvän logiikan~\cite{schmidt_reactor:_1995}.

Reactor-mallin määrittelystä voidaan suoraan huomata, että
sitä ei ole suunniteltu vastaamaan kaikkien
verkkopalvelinsovelluksien tarpeisiin. Kuitenkin
modernissa verkkopalvelinsovelluksessa sen suosioon on
monta syytä. RESTful-rajapinnan käyttäminen
verkkopalvelinsovelluksessa viestien välitykseen,
saa viestien koon pysymään pienenä ja viestien
käsittelyyn kuluvan ajan lyhyempänä verrattuna
HTML-vastaukseen[lähde?]. % Etsi joku REST-lähde
Monisäikeisyys ja rinnakkaistaminen
voidaan järjestelmän arkkitehtuurissa siirtää tietokantamoottorin vastuulle,
jolloin verkkopalvelin voi lähettää sille asynkronisia kyselyitä
ja näin suorittaa useisiin pyyntöihin liittyviä tietokantakyselyitä
samanaikasesti.
Tämänkaltaisella toteutuksella verkkopalvelimen käyttötapaus sopii
hyvin yhteen Reactor-mallin määrittelyn kanssa.

Reactor mallissa pyyntöjen käsittelystä
vastaa tapahtumasäie.
Tapahtumasäie kutsuu jokaista tapahtumaa kohden, sille
määriteltyä tapahtumakäsittelijää.
Tapahtumakäsittelijöissä rinnakaisuuteen päästään
vain I/O:n osalta, kun käytetään asynkronisia
I/O-kutsuja.
Estäviä kutsuja tulee välttää,
sillä niiden käyttäminen laskee sovelluksen
vastaavuutta huomattavasti~\cite{schmidt_reactor:_1995}.
Asynkronisia I/O-kutsuja 
voidaan suorittaa niitä tarjoavien käyttöjärjestelmä kutsujen
avulla, tai siirtämällä I/O:n kutsumisen toisen säikeen tehtäväksi.
Itse Reactor-malli ei ota kantaa miten asynkroninen kutsu toteutetaan.
Jos tapahtumakäsittelijässä tarvitaan pitkäkestoista laskentaa
kannattaa sitä varten luoda uusi prosessi tai säie. Tämä
prosessi tai säie saattaa pyynnön loppuun rinnakkain 
Reactorin tapahtumasäikeen kanssa~\cite{schmidt_reactor:_1995}.

Proactor mallissa on paljon yhtäläisyyksiä Reactor malliin, mutta siinä
hyödynnetään käyttöjärjestelmän asynkronisia ominaisuuksia suorittamaan
operaatiota ennen pyynnön saapumista tapahtumasäikeelle~\cite{pyarali_proactor_1997}.
Tapahtumasäie ohjaa tässä mallissa asynkronisten tapahtumien 
valmistumistapahtumia niitä vastaaville tapahtumankäsittelijöille~\cite{pyarali_proactor_1997}.

Mikään ei kuitenkaan estä luomasta asynkronisten tapahtumien valmistumisille
tapahtumankäsittelijöitä
Reactor-mallissa ja näin tehdään monissa implementaatioissa Future-tai Promise nimisillä
abstraktiolla.

Tässä tutkielmassa keskitytään erityisesti
reactor-malliin, sillä se on sisäänrakennettu Node.js ohjelmaympäristöön ja
täten paljon käytetty. Node.js:n reactor-mallin toteuttamisesta vastaa
libuv-kirjasto~\cite{libuv_design_2019}.

\subsection{Säiereservimalli}
Monisäikeisiä palvelinmalleja on useita, mutta tässä tutkielmassa keskitytään säiereservimalliin.
Monisäikeiset mallit ovat käytännössä syrjäyttäneet moniprosessimallin,
sillä palvelimien käyttöjärjestelmät tukevat säikeitä, ja säikeiden hallinnointiin
kuluu vähemmän resursseja kuin prosessien ja niiden välinen kommunikointi on sujuvampaa.
Moniprosessimallia käytetään kuitenkin yhä, kun järjestelmä pitää
saada skaalautumaan uselle laittestolle.

Monisäikeisissä palvelinsovelluksissa hyödynnetään laitteiston
rinnakkaisuutta tehokkaasti suorittamalla pyyntöjä usealla säikeellä rinnakkain.
Suunnittelun keskeisenä tavoitteena on käsitellä mahdollisimman monta pyyntöä rinnakkain ja 
näin saavuttaa korkea volyymi. Yksittäisen pyynnön
vasteaika saattaa tässä
järjestelmässä kärsiä säikeiden hallinnoimiseen kuluvasta suoritinajasta~\cite{easton_developing_2004}.

Pääsääntöisesti nämä mallit skaalautuvat hyvin laitteistoresursseihin,
mutta jos hallinnoitavien säikeiden määrä on huomattavasti suurempi kuin laitteiston
fyysisten suorittimien määrä, aiheuttaa hallinointi ylimääräistä taakkaa.
Kalleimmat hallinnointitehtävät ovat säikeiden luonti ja tuhoaminen, sillä
muistin varaus- ja vapautusoperaatiot vievät paljon aikaa~\cite{ling_analysis_2000}.
Näihin operaatioihin kuluva aika onkin suurin säie-yhteyttä-kohti mallin
ongelma, jossa jokaista pyyntöä kohti
luodaa uusi säie, joka vastaamisen jälkeen tuhotaan.
Tämän mallin ongelmia on pyritty ratkaisemaan säiereservimallilla.

Säiereservimalli pyrkii minimoimaan tämän ongelman rajoittamalla
säikeiden määrän johonkin järkevään arvoon, kuten fyysisten suorittimien
määrään. Se käyttää samoja säikeitä uudestaan, eli se ei tuhoa säiettä
pyynnön käsittelyn päätteeksi vaan jättää sen vapaaksi säikeeksi reserviin~\cite{ling_analysis_2000}.

Säiereservimallissa kutsujen käsittely tapahtuu seuraavasti.
Kun kutsu saapuu, se osoitetaan vapaalle säikeelle. Säie käsittelee pyyntöä ja,
jos se jää odottamaan I/O:ta asetetaan säikeen tila suorittaa-tilasta odottaa-tilaan,
jolloin käyttöjärjestelmän vuoronantaja voi antaa suoritinaikaa toiselle säikeelle,
kunnes I/O-pyyntö valmistuu. Kun pyyntö on käsitelty ja siihen on vastattu, niin
säie siirretään takaisin reserviin~\cite{ling_analysis_2000}.

Menetelmässä pyyntöjen käsittely on hyvin eristetty toisistaan ja
yhden säikeen lukittava tapahtuma ei vaikuta muihiin säikeisiin negatiivisesti~\cite{davis_case_2017}.
Tämä edistää monisäikeisen mallin vakautta ja kykyä vastustaa hyökkäyksiä.
Tämä malli on myös ohjelmoijalle yksinkertainen sillä, pyyntöjen jakamaa tilatietoa
ja synkronointia on vähän~\cite{hu_applying_1998}.

Säiereservin oikean kapasiteetin valitseminen on erittäin kriittinen 
järjestelmän suorituskyvylle.
Väärin valittu reservin koko kumoaa kaiken hyödyn, mitä luomis
ja tuhoamisoperaatioiden välttämisellä on saavutettu~\cite{ling_analysis_2000}.

Joissain toteutuksissa kokoa voi myös muuttaa dynaamisesti ajon aikana vastaamaan
reaaliaikaiseen tarpeeseen.

\section{Mallit testeissä}
\subsection{Arviontikriteerit}
Palvelimen roolin kannalta olisi toivottavaa, että
se pystyisi käsittelemään mahdollisimman paljon pyyntöjä
ajan yksikössä. Sen tulisi pystyä myös käsittelemään pyyntöjä
luotettavasti eli lopulta kaikki pyynnöt käsitellään ja niihin vastataan.
Yksittäiselle asiakkaalle tärkein suorituskyvyn mittari on se kuinka kauan
yksittäiseen pyyntöön vastaaminen vie. Pyynnön matala vasteaika saa
palvelun käyttökokemuksen tuntumaan sujuvammalta.
Skaalautuvuus on tärkeää palvelinta ylläpitävälle taholle, sillä
näin käyttäjämäärän kasvaessa voidaan suorituskykyä parantaa
lisäämällä laitteistoresursseja. Järjestelmän tulisi myös
olla vastustuskykyinen hyökkäyksille.

Näiden ajatusten ja aikaisemman tutkimuksen perusteella
tutkielmassa vertailtavia järjestelmiä arvioidaan seuraavien
kriteerien valossa~\cite{gokhale_performance_2006}.
\begin{itemize}
    \item käsiteltyjen pyyntöjen määrä eli volyymi
    \item menetysprosentti
    \item skaalautuvuus laitteistoresursseihin
    \item vasteaika
    \item vakaus
\end{itemize}
\subsection{Alustava järjestelmien arviointi kriteerien perusteella}
Molempien mallien puutteita voidaan paikata useamman laitteiston hajautetuilla
järjestelmillä, mutta tälläiset järjestelyt jätetään vertailun ulkopuolelle.

Reactor-mallin järjestelmissä suoritinsidonnaisien
operaatioiden rinnakkaistamiseen ei ole keskitetty erityistä huomiota, ja tälläisten operaatioiden
kanssa järjestelmän käsiteltyjen pyyntöjen määrä ajan yksikössä on keskimäärin heikompi
kuin monisäikeisten järjestelmien~\cite{davis_case_2017}.

Reactor-mallin skaalautuvuus kasvaviin laitteistoresursseihin on hyvin
toteutussidonnaista, kun taas
säiereservimallin skaalautuvuus suoritinresursseihin on hyvin suoraviivaista,
vaikkakin kilpailutilanteiden havaitsemiseen ja estämiseen kuluu enemmän aikaa,
jos säikeita on paljon.

Reactor-mallin palvelimeen voi kohdistaa hyökkäyksiä, jolla pyritään
lukittamaan tapahtumia käsittelevä säie, joka käytännössä vastaa koko ohjelman hallinnasta.
Tähän tarkoitukseen sopivat pyynnöt,
joiden käsittelyyn liittyy kompleksista säännöllisten lauseiden käsittelyä.
Osoitepolkujen selvittämiseen liittyy usein säännöllisiä lausekkeita,
joten verkkopalvelimet ja niiden
kehykset voivat olla altiitta tämänkaltaisille hyökkäyksille~\cite{davis_case_2017}.

Tälläiset hyökkäykset aiheuttaisivat ongelmia myös säiereservimallia
noudattaville järjestelmille, mutta niiden pitäisi onnistua
lukittamaan reservin jokainen säie samaan aikaan, jotta
järjestelmä ei kykenisi enää vastaamaan asiakkaiden pyyntöihin~\cite{davis_case_2017}.

Järjestelmien ominaisuuksien perusteella voimme siis päätellä että,
säiereservimalli
tulisi pärjäämään paremmin, jos palvelimen tehtävän pullonkaula on suoritinaika.
Reactor-järjestelmä
taas luultavasti kykenee pienempään viiveeseen,
jos I/O-kutsujen käsittely on suurin rajoittava tekijä.
\subsection{Järjestelmien staattinen analyysi}

\subsection{Järjestelmien testit simulaatiolla}
Vuonna 1997 James C. Hu et.al testeissään~\cite{hu_measuring_1997}huomasivat
että pienillä tiedostoilla säireservimalli oli paras,
mutta suuremmilla tiedostoilla asynkroninen I/O-kutsujen
tekeminen tapahtumaohjatulla palvelimella oli kaikista tehokkain.
Testissä asynkronisia kutsuja hidasti
erityisesti TransmitFile-funktio, joka oli 
hidas pienillä tiedostoilla.
Tästä tuloksesta voidaan kuitenkin huomata,
että suurilla tiedostoilla saadaan asynkronisilla
kutsuilla aikaan enemmän I/O:n rinnakkaisuutta, sillä
työ siirtyy käyttöjärjestelmän vastuulle.



\section{Johtopäätökset}
%Reactor mallin rajallisuudet rinnakkaisuudessa
%eivät ole haitaksi, jos järjestelmässä on rinnakkaisuutta
%korkeammalla tai matalammalla tasolla.
\bibliography{kandi_zot}
\bibliographystyle{plain}

\end{document}
