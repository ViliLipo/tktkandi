\documentclass[12pt]{article}
\usepackage{cite}

\begin{document}
\section{}
\section{Ongelman esittely}

Palvelimen roolin kannalta olisi toivottavaa, että
se pystyisi käsittelemään mahdollisimman paljon pyyntöjä
ajan yksikössä. Sen tulisi pystyä myös käsittelemään pyyntöjä
luotettavasti eli lopulta kaikki pyynnöt käsitellään ja niihin vastataan.
Yksittäiselle asiakkaalle tärkein suorituskyvyn mittari on se kuinka kauan
yksittäiseen pyyntöön vastaaminen vie. Pyynnön matala vasteaika saa
palvelun käyttökokemuksen tuntumaan responsiivisemmalta.
Skaalautuvuus on tärkeää palvelinta ylläpitävälle taholle, sillä 
näin käyttäjämäärän kasvaessa voidaan suorituskykyä parantaa
lisäämällä laitteistoresursseja. Järjestelmän tulisi myös
olla resilientti hyökkäyksiä vastaan.

Näiden ajatusten ja aikaisemman tutkimuksen perusteella
tutkielmassa vertailtavia järjestelmiä arvioidaan seuraavien 
kriteerien valossa\cite{gokhale_performance_2006}.
\begin{itemize}
    \item käsiteltyjen pyyntöjen määrä
    \item menetysprosentti
    \item skaalautuvuus laitteistoresursseihin
    \item vasteaika
    \item vakaus
\end{itemize}
Asynkronisten ja tapahtumavetoisten palvelinsovelluksien suunnittelun keskiössä
ovat vasteajan pienentäminen ja laitteistoresurssien tehokas hyödyntäminen. 
Tavoitteena on hyödyntää I/O-resursseja mahdollisimman tehokkaasti
rinnakkain ja välttää I/O:n odottamista ohjelman suorituksessa~\cite{pai_flash:_nodate}. Palvelinsovelluksissa
laitteiston verkkoliikennekapasiteetti on usein rajoittava tekijä ja tämän takia
sen tehokas hyödyntäminen on kriittistä. I/O-resurssien asynkronisella hyödyntämisellä on todettu
olevan suorituskykyä parantava vaikutus~\cite{hu_applying_1998}.
Tämän tapaisissa järjestelmissä suoritinsidonnaisien
toimitusten rinnakkaistamiseen ei ole keskitetty erityistä huomiota, ja tälläisten operaatioiden
kanssa järjestelmän käsiteltyjen pyyntöjen määrä ajan yksikössä on keskimäärin heikompi
kuin monisäikeisten järjestelmien~\cite{davis_case_2017}.

Monisäikeisissä palvelinsovelluksissa keskitytään hyödyntämään laitteiston ominaista
rinnakkaisuutta tehokkaasti luomalla jokaisesta pyynnöstä oma säie. Suunnittelun keskeisenä
tavoitteena on käsitellä mahdollisimman monta pyyntöä ajan yksikössä. Yksittäisen pyynnön
vasteaika saattaa tässä
järjestelmässä kärsiä säikeiden hallinnoimiseen kuluvasta suoritinajasta~\cite{easton_developing_2004}.
Menetelmässä pyyntöjen käsittely on hyvin eristetty toisistaan ja
yhden säikeen lukittava tapahtuma ei vaikuta muihiin säikeisiin negatiivisesti~\cite{davis_case_2017}.
Tämä edistää monisäikeisen mallin vakautta ja kykyä vastustaa hyökkäyksiä.


\bibliography{kandi_zot}
\bibliographystyle{plain}

\end{document}
