\documentclass[12pt]{article}
\usepackage{cite}

\begin{document}
\section{}
\section{Ongelman esittely}

Palvelimen roolin kannalta olisi toivottavaa, että
se pystyisi käsittelemään mahdollisimman paljon pyyntöjä
ajan yksikössä. Sen tulisi pystyä myös käsittelemään pyyntöjä
luotettavasti eli lopulta kaikki pyynnöt käsitellään ja niihin vastataan.
Yksittäiselle asiakkaalle tärkein suorituskyvyn mittari on se kuinka kauan
yksittäiseen pyyntöön vastaaminen vie. Pyynnön matala vasteaika saa
palvelun käyttökokemuksen tuntumaan responsiivisemmalta.
Skaalautuvuus on tärkeää palvelinta ylläpitävälle taholle, sillä 
näin käyttäjämäärän kasvaessa voidaan suorituskykyä parantaa
lisäämällä laitteistoresursseja. Järjestelmän tulisi myös
olla resilientti hyökkäyksiä vastaan.

Näiden ajatusten ja aikaisemman tutkimuksen perusteella
tutkielmassa vertailtavia järjestelmiä arvioidaan seuraavien 
kriteerien valossa~\cite{gokhale_performance_2006}.
\begin{itemize}
    \item käsiteltyjen pyyntöjen määrä
    \item menetysprosentti
    \item skaalautuvuus laitteistoresursseihin
    \item vasteaika
    \item vakaus
\end{itemize}
Tutkielmassa vertaillaan tapahtumavetoisia palvelimia monisäikeisiin palvelimiin.
Molempien mallien puutteita voidaan paikata useamman laitteiston hajautetuilla 
järjestelmillä, mutta tälläiset järjestelyt jätetään vertailun ulkopuolelle.
Koejärjestelyissä pyritään simuloimaan riittävällä tarkkuudella todellista
verkkopalvelimen työtaakkaa.

Asynkronisessa ja tapahtumavetoisessa mallissa pyyntöjen alustavasta käsittelystä
vastaa tapahtumasäie. Tapahtumasäie jakaa keskeyttävät operaatiot
työntekijäreservin säikeille. Myös raskasta laskentaa voidaan 
ohjelmoida työntekijäreservin vastuulle. Suurin osa laskentatyöstä
on kuitenkin tapahtumasäikeen vastuulla~\cite{pai_flash:_nodate}.
Asynkronisten ja tapahtumavetoisten palvelinsovelluksien suunnittelun keskiössä
ovat vasteajan pienentäminen ja laitteistoresurssien tehokas hyödyntäminen. 
Tavoitteena on hyödyntää I/O-resursseja mahdollisimman tehokkaasti
rinnakkain ja välttää I/O:n odottamista ohjelman suorituksessa~\cite{pai_flash:_nodate}. Palvelinsovelluksissa
laitteiston verkkoliikennekapasiteetti on usein rajoittava tekijä ja tämän takia
sen tehokas hyödyntäminen on kriittistä. I/O-resurssien asynkronisella hyödyntämisellä on todettu
olevan suorituskykyä parantava vaikutus~\cite{hu_applying_1998}.
Tämän tapaisissa järjestelmissä suoritinsidonnaisien
toimitusten rinnakkaistamiseen ei ole keskitetty erityistä huomiota, ja tälläisten operaatioiden
kanssa järjestelmän käsiteltyjen pyyntöjen määrä ajan yksikössä on keskimäärin heikompi
kuin monisäikeisten järjestelmien~\cite{davis_case_2017}.
Asynkronisten-mallien skaalautumien kasvaviin laitteistoresursseihin on hyvin
implementaatiosidonnaista.

Monisäikeisiä palvelinmalleja on useita, mutta tässä tutkielmassa keskitytään
yksi säie pyyntöä kohti, sekä säiereservimalliin.
Monisäikeisissä palvelinsovelluksissa keskitytään hyödyntämään laitteiston ominaista
rinnakkaisuutta tehokkaasti suorittamalla jokaista pyyntöä omalla säikeellään. Suunnittelun keskeisenä
tavoitteena on käsitellä mahdollisimman monta pyyntöä ajan yksikössä. Yksittäisen pyynnön
vasteaika saattaa tässä
järjestelmässä kärsiä säikeiden hallinnoimiseen kuluvasta suoritinajasta~\cite{easton_developing_2004}.
Pääsääntöisesti nämä mallit skaalautuvat hyvin laitteistoresursseihin,
mutta jos hallinnoitavien säikeiden määrä on huomattavasti suurempi kuin laitteiston
fyysisten suorittimien määrä, aiheuttaa hallinointi ylimääräistä taakkaa.
Tähän ongelmaan on kehitetty säiereservimalli, jossa yhtäaikaa käytössä
olevien säikeiden määrä on rajoitettu esimerkiksi laitteiston fyysisten
suorittimien määrän mukaan.
Menetelmässä pyyntöjen käsittely on hyvin eristetty toisistaan ja
yhden säikeen lukittava tapahtuma ei vaikuta muihiin säikeisiin negatiivisesti~\cite{davis_case_2017}.
Tämä edistää monisäikeisen mallin vakautta ja kykyä vastustaa hyökkäyksiä.
Tämä malli on myös ohjelmoijalle yksinkertainen sillä, pyyntöjen jakama tila
on vähän ja täten myös synkronoinnin tarve pyyntöjen välilä minimoituu~\cite{hu_applying_1998}.

Asynkronisen tapahtumavetoisen palvelimeen voi kohdistaa hyökkäyksiä, jolla pyritään 
lukittamaan tapahtumia käsittelevä säie, joka käytännössä vastaa koko ohjelman hallinnasta.
Tähän tarkoitukseen sopivat pyynnöt,
joiden käsittelyyn liittyy kompleksista säännöllisten lauseiden käsittelyä.
Osoitepolkujen selvittämiseen liittyy usein säännöllisiä lausekkeita,
joten verkkopalvelimet ja niiden
kehykset voivat olla altiitta tämänkaltaisille hyökkäyksille~\cite{davis_case_2017}.

Järjestelmien ominaisuuksien perusteella voimme siis päätellä että,
säie pyyntöä kohden järjestelmä
tulisi pärjäämään paremmin, jos palvelimen tehtävän pullonkaula on suoritinaika.
Asynkroninen järjestelmä
taas luultavasti kykenee pienmpään viiveeseen, jos I/O-kutsujen käsittely on suurin rajoittava tekiä.
\section{Koejärjestely}
Koejärjestelyssä on tärkeää seuraavat seikat

\begin{itemize}
    \item Säädettävä työmäärä pyynnön sisällöllä
    \item Kyky mitata menetysprosentti
    \item mahdollisuus muuttaa laitteistoresursseja?
    \item mahdollisuus mitata vasteaika
\end{itemize}

\bibliography{kandi_zot}
\bibliographystyle{plain}

\end{document}
