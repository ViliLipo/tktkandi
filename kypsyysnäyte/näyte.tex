\documentclass[a4paper, 12pt]{article}
\usepackage[T1]{fontenc}
\usepackage[finnish]{babel}

\author{ Vili Lipo}
\title{Kypsyysnäyte: Rinnakkaisuuden pääperiaatteet}
\date{\today}

\begin{document}

\maketitle
\newpage
% Muodosta   aloitus, käsittely ja lopetus.

% Samanaikaisuus vs. rinnakkaisuus
Termejä samanaikaisuus ja rinnakkaisuus käytetään
usein samoissa asiayhteyksissä vaikka niillä on erilliset merkitykset.
Samanaikaisuus kuvaa järjestelmän ominaisuutta pitää useita
tehtäviä suorituksessa samaan aikaan. Niitä ei kuitenkaan välttämättä
suoriteta samalla hetkellä vaan yhtä prosessia suoritetaan kerrallaan
ja käyttöjärjestelmä jakaa suoritinaikaa siten, että useiden
tehtävien samanaikainen suoritus on ihmisnäkökulmasta mahdollista.
Rinnakkaisuudessa useat konekäskyt ovat
oikeasti suorituksessa täsmälleen samalla hetkellä.
Tämä vaatii rinnakkaisuutta laitteistotasolla.

% Motivaatio
Rinnakkaisuutta tarvitaan ohjelmistojärjestelmien
suorituskyvyn kehittämisessä, sillä yhtä suoritinta
ei voi nopeuttaa rajatta. Rinnakkaisuuden hallinnalla
työtä voi jakaa useille suorittimelle ja näin 
parantaa ohjelmistojärjestelmän suorituskykyä.

% Rakenne
Rinnakkaisuuden voi mieltää ilmentyvän tietokonejärjestelmässä
kolmella tasolla. Nämä tasot ovat laitteisto-, käyttöjärjestelmä,-ja sovellustasot.
Tasojen rinnakkaisuus rakentuu päällekkäin, sillä käyttöjärjestelmän rinnakkaisuus
on riippuvainen laitteiston rinnakkaisuudesta ja taas sovelluksien rinnakkaisuus
on riippuvainen käyttöjärjestelmän rinnakkaisuudesta.

% Laitteiston kuvaus
Jotta rinnakkaisuutta voisi olla järjestelmän sovelluksissa tai käyttöjärjestelmässä,
pitää laitteistossa olla useita suorittimia.
Myös usean laitteen hajautetuissa järjestelmissä kyseessä on rinnakkaisuus.
Rinnakkaista laskentaa suoritetaan usealla laitteella
esimerkiksi palvelinkeskuksissa.


Nykyisissä suoritinarkkitehtuureissa on yhdellä sirulla useita suorittimia,
joita kutsutaan ytimiksi. Jokaisessa ytimessä on kaksi laitteistotason
säiettä.
Laitteistotason säikeet on toteutettu kahdentamalla suorittimen rekisterit.
Kun toinen ytimen säikeistä päätyy tilanteeseen, jossa vaaditaan muistinouto, suoritin
vaihtaa käyttöön toisen säikeen rekisterit ja jatkaa sen suorittamista 
muistinoudon rinnalla. Hidas muistinouto-operaatio 
ei siten tuhlaa kellopulsseja.
Laitteisto esittää käyttöjärjestelmälle jokaisen säikeen loogisena suorittimena.
% Käyttöjärjestelmän kuvaus

Käyttöjärjestelmäytimen tehtävä on olla rajapinta laitteiston ja muiden ohjelmistojen 
välillä, siksi on luonnollista, että laitteiston rinnakkaisuudesta vastaaminen
on käyttöjärjestelmän vastuulla. Käyttöjärjestelmät ovat jo hyvin pitkään
aikatauluttaneet useita prosesseja ja ydintason säikeitä samanaikaisesti myös 
järjestelmissä, joissa on 
vain yksi suoritin.
Ydintason säikeet ovat käyttöjärjestelmän konsepti, jotka
eivät suorasti liity laitteiston säikeisiin. 
Jatkossa
säikeillä tarkoitetaan ohjelmistotason konseptia 
eikä laitteistotason toteutusta.

%prosessi ja säikeet

Prosessit ja säikeet kuvastavat ajossa olevia
ohjelmia järjestelmässä. Ne sisältävät tietorakenteet,
joita käyttöjärjestelmä tarvitsee ohjelmien hallinnointiin.
Prosessissa on yksi tai useampia säikeitä.
Säikeet voivat olla käyttäjätason tai ydintason. Käyttäjätason 
säikeiden aikataulutuksesta vastaa käyttäjätason ohjelmisto
esimerkiksi säiekirjasto tai muu kehysympäristö.
Käyttöjärjestelmä ei ole tietoinen käyttäjätason säikeistä
ja täten ei voi aikatauluttaa niitä, joten saman
prosessin käyttäjätason säikeet eivät voi olla
suorituksessa rinnakkain. 
Käyttöjärjestelmä jakaa ydintason säikeille suoritinaikaa samaan
tapaan kuin prosesseille.

% Vuoronanto

Yksiprosessorijärjestelmissä käyttöjärjestelmä vuorottelee
ajossa olevien säikeiden välillä. Prosessit
ovat siis näennäisesti samanaikaisesti suorituksessa,
mutta todellisuudessa käyttöjärjestelmä suorittaa
niitä vuorotellen peräkkäin. Yksittäisen prosessin
suoritus voi keskeytyä missä vain kohtaa suoritusta, 
ja käyttöjärjestelmä vaihtaa toisen prosessin suoritukseen.
Prosessi palaa suoritukseen vuorollaan.
Prosessien suoritusjärjestys ei ole ennalta määrätty, mutta
käyttöjärjestelmän aikataulutusalgoritmiin liittyy yleensä
tapoja priorisoida tiettyjä prosesseja.

% Rinnakkaisuus käytännössä

Kun laitteistossa on useita suorittimia, käyttöjärjestelmä
voi aikatauluttaa prosesseja rinnakkain niillä kaikilla.
Käyttöjärjestelmässä tapahtuu yhä
vuorottelua prosessien välillä, sillä normaalissa
järjestelmässä suoritettavien prosessien määrä on suurempi
kuin laitteiston suorittimien määrä.

Sovelluksissa voidaan hyödyntää käyttöjärjestelmän
rinnakkaisuutta suorittamalla sovelluksen osia useilla säikeillä.
Rinnakkaisuuden jäsentämiseen sovellustasolla on useita
menetelmiä. 
Hajota ja hallitse tyylisiä 
algoritmeja voi jakaa usean säikeen
toteutettavaksi.
Suosittu suunnittelumalli on myös
tuottaja-kuluttaja malli, jossa tuottaja täyttää
puskuriin käsittelemäänsä tietoa, josta kuluttaja siirtää
sen jatkokäsittelyyn.

% Ongelmien kuvaus

Samanaikaisuuden ja rinnakkaisuuden hallintaan liittyy kekseisiä ongelmia
kuten kilpailutilanteet ja lukkiutumiset.
Kilpailutilanteet ovat ohjelmointivirheitä,
joissa oletetaan, että eri säikellä tehtävä työ
valmistuu tietyssä järjestyksessä. Säikeiden
aikataulutus on kuitenkin käyttöjärjestelmän vastuulla
ja täten muiden ohjelmien näkökulmasta lähes mielivaltainen.
Ohjelman tulisikin tuottaa oikea tulos
kaikissa mahdollisissa suoritusjärjestyksissä.
Kilpailutilanne voi olla vaikeaa löytää
ohjelmasta, koska virheen syntyminen
on riippuvainen käyttöjärjestelmästä eikä käytetystä syötteestä.

% Lukkiutuminen

Lukkiutuminen on järjestelmän resurssien varaamiseen liittyvä
ongelma. Lukkiutumisessa kaksi tai useampia prosesseja
tarvitsee samoja resursseja. Jos resurssien varausjärjestys
on epäonninen yksi prosessi on varannut resurssin mitä toinen
tarvitsee, mutta toinen on varannut jo resurssin mitä ensimmäinen
tarvitsee seuraavaksi. Nyt molemmat pitävät yhtä resurssia ja odottavat toista.
Tästä seuraa lukkiutuminen.
Lukkiutumisen purkaminen vaatii käyttöjärjestelmän väliintuloa,
koska prosesseilla ei ole valtaa puuttua toisen prosessin resursseihin.

Rinnakkaisuus ohjelmistojärjestelmässä on 
toivottavaa suorituskyvyn kannalta.
Se kuitenkin esittää useita haasteita
ja sudenkuoppia kehitystyölle.
Koska yksittäisen suorittimen nopeuden kasvattaminen
on ollut jo vuosia
haastavaa, suoritinvalmistajat keskittyvät mahduttamaan
yhä enemmän ytimiä yhdelle sirulle. Rinnakkaisuuden merkitys
ohjelmistojen suorituskyvylle tulee tämän perusteella kasvamaan entisestään.
Sovelluskehittäjän onneksi ohjelmointikielisässä ja kehysympäristöissä
on jo pitkään kehitetty rinnakkaiseen ohjelmointiin
sopivia tietorakenteita ja toimintoja.





\end{document}
